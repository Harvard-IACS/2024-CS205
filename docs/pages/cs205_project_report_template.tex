\documentclass[11pt]{article}

\usepackage[letterpaper,includeheadfoot,top=1cm,bottom=2cm,left=2cm,right=2cm]{geometry}
\usepackage{here}
\usepackage{amsmath}
\usepackage{amssymb}
\usepackage{mathtools}
\usepackage{bm}
\usepackage{sectsty}
\usepackage{graphicx}
\usepackage[colorlinks=true,urlcolor=blue]{hyperref}
\usepackage{authblk}
\usepackage{caption}
\usepackage{fancyvrb}
\usepackage{relsize}

\title{This is the title of the project report}
\author[1]{First author}
\author[2]{Second author}
\author[1,2]{Third author}
\author[1,2]{Fourth author}
\affil[1]{Affiliation of first author}
\affil[2]{Other affiliation}

\date{\today}

\renewcommand\thesubsection{}


\begin{document}
\maketitle

\begin{abstract}
  TODO: This is the abstract of the proposal report: in this document we
  provide a template with the required sections.  Additional guidelines are
  provided for selected sections below. The project report should be in the
  range of \textbf{8-10 pages} including graphs and references, and must
  contain the following information:
  \begin{itemize}
    \item Abstract
    \item Background and significance
    \item Scientific goals and objectives
    \item Algorithms and code parallelization (including memory requirements)
    \item Representative performance benchmarks and scaling analysis
      (strong and weak scaling)
    \item Resource justification
  \end{itemize}
  The report should be typeset using \LaTeX{}\cite{lamport1993a}.  You can use
  this template as a baseline.
\end{abstract}

\section{Background and Significance}

  TODO: In this section, describe the problem you are going to solve, the
  research methods, its significance and a comparison with existing work.


\section{Scientific Goals and Objectives}

  TODO: What are your scientific goals and objective with this project?
  Justify the need for compute hours on a HPC architecture.


\section{Algorithms and Code Parallelization}

  TODO: Please insert in this section a description of the methods and
  algorithms of all the codes adopted for your computational study, justifying
  your choices and describing possible alternatives. Furthermore, please
  specify whether you are the main developer, a contributing developer or a
  user of the code. You should include a brief list of the main scientific
  libraries employed (if any) and a description of the parallelization
  approach, with specific memory and I/O requirements as well.

  Report if the code employs OpenMP shared memory parallelism, MPI distributed
  parallelism or hybrid MPI/OpenMP, which type of MPI communication has been
  implemented and if it makes use of shared memory parallelism, GPU
  accelerators or OpenACC/CUDA (optional), with specific memory and I/O
  requirements as well.

  \subsection{Validation, Verification}

  Please explain how to validate your method against experiments or other
  established reference data and verify the numerical consistency of your
  model, citing the relevant references to peer reviewed papers.


\section{Performance Benchmarks and Scaling Analysis}

  TODO: Please report in this section the results of the mandatory performance
  analysis of your code (roofline analysis) as well as strong and weak scaling
  tests.  You should report scaling data (tables) and plots for the code of
  your project. 

  You should select meaningful job sizes to simulate the representative
  systems, compatible with reasonably short runtimes: the lowest number of
  nodes is determined in general by memory and wall time constraints, while the
  highest node counts should let you identify the job size at which you reach
  $\sim 50\%$ of the parallel efficiency with respect to ideal scaling.

  You are also requested to provide some key figures of the jobs that you 
  ran during the development of the project:
  \begin{itemize}
    \item the wall clock time of your typical submission in hours
    \item the expected job size of your typical job in number of nodes
    \item the memory per node requested by the typical job in GB
    \item the maximum number of input files read by a job
    \item the maximum number of output files written by a job
    \item the library used for I/O (if any)
  \end{itemize} 
  Examples for I/O libraries are: HDF5, NetCDF or MPIIO.

  Table~\ref{table:workflow_parameters} provides a template of how these 
  figures should be presented in the report. 
  The given example describes two test cases used for the development presented in two
  columns.  Note that the given numbers are dummy data, the typical wall clock
  time for your tests in this project should be less than what is shown here.
  \begin{table}[H]
    \begin{center}
      \begin{tabular}{@{}*3{r}@{}}
        \hline \hline
        & Test case A & Test case B \\
        \hline \hline
        Typical wall clock time (hours) & 6 & 12 \\
        Typical job size (nodes) & 16 & 16 \\
        Memory per node (GB) & 12 & 18 \\
        Maximum number of input files in a job & 4 & 6 \\
        Maximum number of output files in a job & 8 & 14 \\
        Library used for I/O & HDF5 & HDF5 \\
        \hline \hline
      \end{tabular}
    \end{center}
    \caption{Workflow parameters of the two test cases used during project
    development.}
    \label{table:workflow_parameters}
  \end{table}


\section{Resource Justification}

  TODO: The request of the annual amount of node hours should be clearly linked
  with the node hours used by the representative benchmarks: the number of node
  hours consumed by a simulation is computed multiplying the number of nodes by
  the wall time expressed in hours for a typical production run.

  For example, the optimal job size of
  the representative benchmark is 16 nodes.  Assuming the corresponding wall time
  for a production run is 141~s, which is then equivalent to $\sim
  0.6267$ node hours, as a result of the following product:
  \[
  0.6267\text{ node hours} = 16\text{ nodes}\times\frac{141 s}{3600 \frac{s}{\text{hour}}}
  \]
  The benchmark is short and represents in general a small number of iterations
  (cycles, timesteps or an equivalent measure), while in a real production
  simulation we will need to extend it.

  Therefore we will estimate how many iterations should be necessary to
  complete a simulation in production. Furthermore, your project might contain
  multiple runs to complete a task, each of them requiring several sets of
  simulations to complete.  The total resource request will sum the
  corresponding node hours obtained multiplying all the factors reported in
  Table~\ref{table:resource_request}. 
  \begin{table}[H]
    \begin{center}
      \begin{tabular}{@{}*3{r}@{}}
        \hline \hline
        & Test case A & Test case B \\
        \hline \hline
        Simulations per task & 2 & 4 \\
        Iterations per simulation & 5000 & 10000 \\
        node hours per iteration & 0.6267 & 0.6267 \\
        Total node hours & 6267 & 25068 \\
        \hline \hline
      \end{tabular}
    \end{center}
    \caption{Justification of the resource request}
    \label{table:resource_request}
  \end{table}
  The small example above will request a total of 31335 node hours to cover
  both test cases in the proposal to the supercomputing center.  Note that the
  simulations listed in 
  Table~\ref{table:resource_request} must usually be motivated by a project
  plan that accommodates the proposal.  We will omit this section for the
  purpose of the CS205 project.


  % you may use thebibliography shown here or use a bibliography file
  % 'main.bib' as in the commented example below.
  \begin{thebibliography}{9}
    \bibitem{lamport1993a} 
      Leslie Lamport. 
      \textit{\LaTeX{}: a document preparation system}.
      Addison-Wesley, Reading, Massachusetts, 1993.
  \end{thebibliography}

  % \section*{References}
  % \bibliographystyle{abbrv}
  % \bibliography{main} % the file main.bib must exist in your document folder.

\end{document}

% vim-local: LaTeX
